\documentclass{article}

\usepackage{graphicx}
\usepackage[utf8]{inputenc}
\usepackage[T1]{fontenc}
\usepackage[portuguese]{babel}

\newcommand{\tit}[1]{\textit{#1}}

\begin{document}

\title{Gerador de FFT (Fast Fourier Transform) em Tempo Real}
\author{Esdras R. Carmo, Gabriel R. Hioki}

\maketitle
\newpage
\tableofcontents
\newpage

\section{Introdução}
A Transformada de Fourier Rápida (FFT) é um algoritmo que recebe um sinal
$x_n$ de amostras no domínio do tempo e retorna o vetor $X_m$ com os coeficientes
das funções senoidais em diferentes frequências. Em outras palavras, o FFT irá 
transformar um sinal no domínio do tempo para o domínio da frequência.

O FFT realiza otimizações no básico DFT (Discrete Fourier Transform) que consiste na seguinte
transformação linear:
$$
    X_m = \sum_{n=0}^{N-1}x_n w^{nm}
$$
onde $N$ é o tamanho dos vetores, $w = e^{2i\pi/N}$ são os fatores de rotação (\tit{twiddle}) e
$0 \leq m < N$. O algoritmo básico do somatório do DFT é feito usando $N^2$ operações, enquanto
o FFT consegue realizar o cálculo com $N \log N$ operações \cite{fft-hardware}.

\section{Diagrama de Blocos}

\section{Descrição Funcional}

\newpage
\bibliography{doc}
\bibliographystyle{ieeetr}
\end{document}
